\section{Giá trị thực tiễn}
Fichain mở ra các mô hình kinh doanh mới, tối ưu hóa hoạt động hiện tại và tạo nền tảng cho các hệ sinh thái Web3 trong lĩnh vực tài chính - ngân hàng và kinh tế số.

\subsection{Đối với Ngân hàng và Tổ chức Tài chính}
\begin{itemize}
    \item \textbf{Số hóa Tài sản \& Chứng khoán (Asset Tokenization):} Cho phép phát hành, quản lý và giao dịch các loại tài sản được mã hóa như trái phiếu doanh nghiệp, chứng chỉ quỹ, bất động sản một cách minh bạch, hiệu quả và giảm thiểu chi phí phát hành.
    \item \textbf{Thanh toán Xuyên biên giới:} Xây dựng hệ thống chuyển tiền và thanh toán quốc tế tức thì, chi phí thấp, giảm sự phụ thuộc vào các mạng lưới trung gian truyền thống (như SWIFT\cite{swift_website}, VISA\cite{visa_website}) và tăng tốc độ thanh khoản.
    \item \textbf{Tài chính Doanh nghiệp \& Tín dụng:} Tạo lập các nền tảng cho vay ngang hàng (P2P Lending), tài trợ thương mại (Trade Finance), quản lý chuỗi cung ứng trên blockchain, tăng cường tính minh bạch và hiệu quả trong việc thẩm định và giải ngân.
    \item \textbf{Hiện đại hóa Core Banking \& Liên kết Web3:} Giảm tải cho các hệ thống core banking kế thừa bằng cách chuyển các nghiệp vụ mới lên một nền tảng linh hoạt, bảo mật và dễ tích hợp với các dịch vụ Web3, smart contract và các ứng dụng phi tập trung (DApp), đồng thời mở ra khả năng tương tác với các hệ thống ngân hàng kỹ thuật số trong và ngoài nước.
\end{itemize}

\subsection{Đối với Nền kinh tế và Xã hội}
\begin{itemize}
    \item \textbf{Thúc đẩy tài chính toàn diện:} Giảm chi phí giao dịch và tạo ra các sản phẩm tài chính mới, dễ tiếp cận hơn cho người dân và doanh nghiệp nhỏ.
    \item \textbf{Nâng cao năng lực công nghệ quốc gia:} Việc làm chủ một công nghệ nền tảng như blockchain Layer 1 không chỉ khẳng định vị thế công nghệ của Việt Nam, mà còn cho phép tích hợp sâu với các dịch vụ công của chính phủ như định danh số (eKYC), quản lý tài sản công, lưu trữ hồ sơ y tế và giáo dục, góp phần xây dựng chính phủ số hiệu quả và minh bạch.
\end{itemize}
