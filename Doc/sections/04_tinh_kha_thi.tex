\section{Tính khả thi và khả năng mở rộng}

\subsection{Tính khả thi kỹ thuật}
Tính khả thi kỹ thuật của Fichain được đảm bảo bởi năng lực và kinh nghiệm của đội ngũ phát triển. Đội ngũ bao gồm các chuyên gia hàng đầu về:
\begin{itemize}
    \item \textbf{Phát triển Blockchain Layer 1:} Có kinh nghiệm thiết kế, xây dựng, tối ưu và vận hành mainnet cho blockchain Layer 1 với hiệu suất cao (hàng nghìn TPS).
    \item \textbf{Phát triển DApps và Smart Contract:} Am hiểu sâu sắc về EVM, Solidity, và các tiêu chuẩn token phổ biến.
    \item \textbf{Core Banking và Tích hợp hệ thống:} Có kinh nghiệm thực tế trong việc tích hợp các hệ thống ngân hàng lõi với các hệ thống vệ tinh, tuân thủ các giao thức và quy định của ngành.
    \item \textbf{Khoa học dữ liệu và AI trong tài chính:} Có khả năng xây dựng các mô hình AI/ML cho các bài toán như đánh giá rủi ro, phát hiện gian lận.
\end{itemize}
Sự kết hợp này khẳng định đội ngũ có đủ năng lực để xây dựng, triển khai và hỗ trợ vận hành Fichain.

\subsection{Tính khả thi tài chính và mô hình kinh doanh}
Fichain lựa chọn mô hình kinh doanh B2B với định hướng hợp tác linh hoạt và đa dạng, phù hợp với nhiều loại hình doanh nghiệp. Các hình thức hợp tác bao gồm:
\begin{itemize}
    \item \textbf{Phí giấy phép sử dụng nền tảng (Licensing Fee)}: Áp dụng cho các đối tác có nhu cầu tích hợp và triển khai nền tảng Fichain vào hệ thống riêng.
    \item \textbf{Phí giao dịch (Transaction Fee)}: Thu phí trên mỗi giao dịch được thực hiện qua nền tảng.
    \item \textbf{Mô hình chia sẻ doanh thu (Revenue Sharing)}: Thỏa thuận chia sẻ lợi nhuận linh hoạt, dựa trên quy mô và đặc thù của từng dự án hợp tác.
\end{itemize}
Cấu trúc doanh thu này giúp Fichain đảm bảo tính ổn định và bền vững về tài chính, đồng thời tạo điều kiện thuận lợi để mở rộng hệ sinh thái và đầu tư vào phát triển công nghệ lâu dài.

\subsection{Tính khả thi pháp lý}
Fichain được thiết kế với sự tuân thủ pháp lý là cốt lõi. Nền tảng cho phép:
\begin{itemize}
    \item Triển khai dưới dạng mạng riêng tư (private) hoặc liên minh (consortium), cho phép kiểm soát hoàn toàn danh tính các bên tham gia.
    \item Tích hợp các quy trình KYC/AML.
    \item Phân quyền truy cập dữ liệu giao dịch, đáp ứng yêu cầu của cơ quan quản lý khi cần thiết.
\end{itemize}

\subsection{Khả năng mở rộng (Scalability)}
Nền tảng Fichain được thiết kế để có khả năng mở rộng cao ngay từ đầu, với mục tiêu xử lý hàng nghìn giao dịch mỗi giây (TPS) nhờ cơ chế đồng thuận PoSA hiệu quả. Kiến trúc này đủ sức đáp ứng nhu cầu giao dịch của các ứng dụng tài chính quy mô lớn.
