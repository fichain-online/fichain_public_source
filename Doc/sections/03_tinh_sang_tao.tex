\section{Tính sáng tạo}
Fichain khác biệt so với các giải pháp hiện có nhờ vào sự kết hợp của các yếu tố độc đáo, được thiết kế chuyên biệt.

\begin{itemize}
  \item \textbf{Lựa chọn Native Coin thực tiễn, không đầu cơ:} Trong khi hầu hết các blockchain Layer 1 tạo ra một native coin mới với giá trị biến động và mang tính đầu cơ (như ETH\cite{buterin2014ethereum}, SOL\cite{yakovenko2017solana}, AVAX\cite{rocket2019scalable}, ATOM\cite{kwon2016cosmos}), Fichain có một lựa chọn thiết kế mang tính đột phá và thực tiễn: \textbf{sử dụng Việt Nam Đồng làm đơn vị tiền tệ gốc}. Cách tiếp cận này loại bỏ rào cản về tâm lý và rủi ro biến động giá, trực tiếp giải quyết mối quan tâm hàng đầu của các tổ chức tài chính và cơ quan quản lý.

  \begin{itemize}
        \item Cơ chế đồng thuận \textbf{PoSA} yêu cầu các validator phải là những định chế tài chính được cấp phép và có định danh (Authority), đảm bảo chỉ những bên đáng tin cậy mới có quyền tham gia xác thực giao dịch.
        \item Nền tảng cho phép thiết lập các \textbf{mạng riêng tư (private) hoặc liên minh (consortium)}, toàn quyền kiểm soát việc truy cập và chia sẻ dữ liệu.
        \item Khả năng tích hợp sẵn các quy trình \textbf{KYC/AML} ở cấp độ giao thức, đảm bảo mọi giao dịch đều có thể truy vết và tuân thủ pháp luật.
  \end{itemize}
    

    \item \textbf{Hạ tầng tương thích Web2 và mở rộng sang Web3:} Fichain được thiết kế để có thể tích hợp dễ dàng với các hệ thống tài chính truyền thống thông qua API chuẩn hóa, đồng thời mở ra khả năng kết nối với các ứng dụng Web3 hiện đại. Điều này giúp các ngân hàng và tổ chức tài chính từng bước chuyển đổi mà không bị gián đoạn, tận dụng cả hạ tầng hiện tại và tiềm năng phi tập trung trong tương lai.

    \item \textbf{Khả năng tích hợp dịch vụ công và định danh số:} Không chỉ phục vụ khối ngân hàng, Fichain còn hướng tới khả năng tích hợp với các dịch vụ công như định danh điện tử (eKYC), quản lý hồ sơ thuế, hồ sơ công dân, và các dịch vụ hành chính số. Điều này mang lại một kiến trúc blockchain linh hoạt, có thể phục vụ cả khối tư nhân lẫn nhà nước trong hệ sinh thái tài chính số toàn diện.

  \item \textbf{Kiến trúc Cầu nối An toàn giữa Tài chính Truyền thống và Tương lai Số:}
    Fichain không đặt mục tiêu thay thế hoàn toàn hệ thống core banking, mà định vị mình là một "cầu nối" chiến lược và an toàn. Nền tảng cho phép các ngân hàng giữ nguyên hệ thống lõi ổn định của mình, đồng thời sử dụng Fichain như một "sandbox" đổi mới để:
    \begin{itemize}
        \item \textbf{Thử nghiệm và triển khai} các sản phẩm tài chính số (số hóa tài sản, DeFi) một cách nhanh chóng và ít rủi ro.
        \item \textbf{Kết nối} một cách an toàn giữa dữ liệu trong hệ thống kế thừa và các ứng dụng trên blockchain.
        \item \textbf{Từng bước hiện đại hóa} các dịch vụ mà không gây gián đoạn cho hoạt động kinh doanh cốt lõi. Đây là một lộ trình chuyển đổi số thực tế và khả thi cho các tổ chức tài chính lớn.
    \end{itemize}

\end{itemize}
